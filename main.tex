%\documentclass{article}
\documentclass[conference]{IEEEtran}
\IEEEoverridecommandlockouts
% The preceding line is only needed to identify funding in the first footnote. If that is unneeded, please comment it out.
\usepackage{cite}
\usepackage{amssymb,amsfonts}
\usepackage{algorithmic}
\usepackage{textcomp}
\usepackage[utf8]{inputenc}
\usepackage{amsmath}
\usepackage{graphicx}
\usepackage{xcolor}
\usepackage{accents}
\usepackage{array}
\usepackage{balance}
\usepackage{enumitem}
\setlist[itemize]{leftmargin=4.99mm}
%\usepackage[
%backend=biber,
%style=apa,
%citestyle=apa
%]{biblatex}

%\addbibresource{bibliography.bib} %Imports bibliography file
\begin{document}

%\title{User-centric and Context-aware  Recommender System for e-Tourism}

%\title{User-centric and Context-aware  Recommender System for e-Tourism}
\title{
%ReCeSO o ReCREO: User-centric and Context-aware  Recommender System for e-Tourism:
RECESO: Recommender System for E-tourism with Serendipity and Ontology-based }

\author{\IEEEauthorblockN{Gustavo Castellanos}
\IEEEauthorblockA{\textit{Computer Science Department} \\
\textit{Universidad Simón Bolívar}\\
Caracas, Venezuela \\
14-10192@usb.ve}
\and
\IEEEauthorblockN{Yudith Cardinale}
\IEEEauthorblockA{\textit{Computer Science Department} \\
\textit{Universidad Simón Bolívar}\\
Caracas, Venezuela \\
ycardinale@usb.ve}
\and
\IEEEauthorblockN{Philippe Roose}
\IEEEauthorblockA{\textit{Université de Pau et des Pays de l’Adour} \\
\textit{LIUPPA -- T2I64600}\\
Anglet, France \\
philippe.roose@iutbayonne.univ-pau.fr}
%\and
%\IEEEauthorblockN{Yudith Cardinale}
%\IEEEauthorblockA{\textit{Computer Science Department} \\
%\textit{Universidad Simón Bolívar}\\
%Caracas, Venezuela \\
%ycardinale@usb.ve}
%\and
%\IEEEauthorblockN{Marc Dalmau}
%\IEEEauthorblockA{\textit{Université de Pau et des Pays de l’Adour} \\
%\textit{LIUPPA -- T2I64600}\\
%Anglet, France \\
%dalmau@iutbayonne.univ-pau.fr}
%\and
%\IEEEauthorblockN{Nadine Couture}
%\IEEEauthorblockA{\textit{ESTIA Technopole Izarbel} \\
%\textit{64210}\\
%Bidart, France \\
%n.couture@estia.fr}
%\and
%\IEEEauthorblockN{Dominique Masson}
%\IEEEauthorblockA{\textit{dept. name of organization (of Aff.)} \\
%\textit{name of organization (of Aff.)}\\
%City, Country \\
%email address or ORCID}
}


%\author{Gustavo Castellanos and Yudith Cardinale and Philippe Roose }
%\date{November 2019}


\maketitle

\begin{abstract}
Often travelers try to manually collect enough information for planing to go to a destination or when being at this destination, but this task may be overwhelming due to the big amount of options (including local apps download). Usually, tourists depend on places' reviews to make the choice, but this implies prior knowledge of the existence of the places. These reviews are often requested on tourism sites and applications, most of which follow the PULL paradigm, i.e., the user has to explicitly search for suggestions/recommendations through interaction with the application's graphical user interface. On the other hand, a
PUSH approach, in which the application proactively triggers a recommendation process when
necessary -- for example, when the user is close to a point of interest  (POI), when lunch time is approaching, or
when weather conditions change -- seems to be a more reasonable solution. 
%Having said that, 
In this context, 
%in this work 
we propose a user-centric hybrid recommender system, using an ontology-based algorithm for content-based and context-aware recommendations, with a focus on decreasing predictable output under same situations, i.e., increasing serendipity based on a {\it aging-like} approach. To demonstrate its suitability and performance, we test it in different simulated scenarios using different weights or priorities for the predicted preferences, context activation, aging, and distances of the recommended places. Results show how the spreading activation and the proposed aging system give relevant and varied recommendations for the users, while taking into account their context, and how the system could be configured to focus more on either relevant, varied, context-activated, or near places.
\end{abstract}
\begin{IEEEkeywords}
tourism, recommender systems, context awareness
\end{IEEEkeywords}


\section{Introduction}
%\textcolor{red}{XXXLa intro y el abstract deben tener la siguiente estructura: hablar del contexto general: Turismo y e-tourismo dado los avances tecnológicos; lo que se usa y sus limitaciones (que es el párrafo que tienes ahora); luego lo que se propone (la contribución del paper, la conclusión general de los resultados y finalmente la organización del paperXXXX}

%\textcolor{red}{XXX Pondré en rojo XXXXMis comentariosXXXXXXXX y los párrafos que redacte/cambie. Así los ubicas más fácilmente}

Tourism is one of the most promising areas for diverse applications and is becoming an extremely
important market \cite{buhalis2011tourism,murphy2013tourism,fermoso2015open,ku2015cultivating,alghamdi2016tourism,artemenko2017tourism,kazandzhieva2019tourism}. 
When travellers are going or planning to go to a destination, they try to collect information about their new destination “as best they can” (e.g., by going to the tourist office or by obtaining information about their destination and its surroundings on the Internet). Usually, tourists depend on places' reviews to decide where to spend their vacations, their free time, or even their lunch or just their afternoon, but this requires prior knowledge of the existence of the places. Furthermore, sometimes people know about a place but do not know if it is a good fit for their necessities and ignores it, taking the risk of missing a good experience. Nevertheless, this requires a considerable amount of effort on the part of the tourists.

Currently, there is a plethora of applications offering information on cities and historical centres, points of interest (POIs) to visit, city tours, green areas to rest, etc. The use of such applications implies that the user is first aware of them, then installs them in the hope that they are suitable (only 10\% of downloaded mobile applications are used more than once), learns and knows how
to use them and then filters the data/information that is uploaded or offered. Most of these
applications follow the PULL paradigm, i.e., the user has to explicitly search for
suggestions/recommendations through interaction with the application’s graphical user interface.
We believe that this is one of the reasons why these types of applications are not successful: they
require too many steps, too many interactions, and too many unknowns. On the other hand, a
PUSH approach, in which the application proactively triggers a recommendation process when
necessary (for example, when the user is close to a POI, when lunch time is approaching, or
when weather conditions change) seems to be a more reasonable solution. A first step in this
direction has been taken by Google via Google Now\footnote{https://www.google.com/intl/fr/landing/now/}, 
which provides information/suggestions
to users by detecting in their geographical environment what they need according to their
location and time. However, such applications are simplistic and use only a limited amount of
information, and moreover, do not take into account the user’s activity. A key element is the
consideration of the users' context, such as their locations, weather (\textit{is it raining?}), or time (\textit{is it night?}).

Recommender Systems are now at the heart of
much research and offer real benefits to users, organizations, and the business community in
general~\cite{borras2014intelligent,del2016pull,leskovec2020mining}. In the context of tourism, most recommender systems are able to predict user's preferences based on previous activities, but research on Context-Aware Recommender Systems
(CARS) are still to be further developed~\cite{adomavicius2011context,nejma2015service,haruna2017context,raza2019progress}. Additionally, most recommender systems do not consider to vary the recommendations -- i.e., they become predictable  for same users, under the same situations. Thus, there is still a real need for research and design of innovative solutions in this field. Nowadays, the use of ontologies to represent the knowledge related to tourism (POI, users' information, context information, etc.) is becoming a powerful tool to offer such as innovative solutions \cite{rajaonarivo2019rec} \cite{bahramian_abbaspour_claramunt_2017}, being DATAtourisme\footnote{http://info.datatourisme.gouv.fr/ontology/core/2.0/} a good example.

In this context, in this work we propose a user-centric hybrid recommender system, using an ontology-based algorithm for content-based and context-aware recommendations. According to the users’ wishes, the serendipity of recommendations may increase (e.g., surprise events, unrepeated recommendations)~\cite{kotkov2016survey}. All this based on an {\it aging}-like algorithm, which gives less priority to recently recommended places.  It is also desirable that the user knows why something is recommended (i.e., that the recommendations are explainable). We describe the architecture and algorithms of our recommender system. To demonstrate its suitability and performance, we test it in different simulated scenarios using different weights or priorities for the predicted preferences, context activations, aging and distances of the recommended places.

We organize the rest of this paper as follows. In Section \ref{section:related-work}, we survey 
%introduces 
relevant studies related to our work.
%and concepts for understanding our system. 
Section \ref{sec:proposal} describes the architecture and functioning of our proposed system. Section \ref{section:study-case} describes results obtained during the experiments. Finally, Section \ref{section:conclu} explains our final thoughts.



\section{Related Work}
\subsection{Recommender Systems}
\subsection{E-Tourism}
\subsection{Spreading Activation}
\cite{bahramian_abbaspour_claramunt_2017} integrate the concepts of recommender systems with semantic web at building a context-aware tourism recommender system based on the \textit{spreading activation} algorithm. In their own words: "\textit{In spreading activation method, a given concept is represented by a node and has an activation value. A relation among different concepts is represented by a link between nodes and has a weight value. To initialize the algorithm, one or several nodes of a network are activated and these activations spread to the relevant nodes. This process is iterated until a stopping condition (e.g., number of node processed) is reached}". The activation value of the node \(v_i\) is computed as follows:
\begin{equation} \label{eq:og_activation}
a_{v_i} = \sum_{v_j \in neighbors(v_i)} w_{v_i, v_j} a_{v_j} 
\end{equation}
where $a_{v_i}$ is the activation value of node $v_i$, $neighbors(v_i)$ is the set of $v_i$'s neighbor nodes and $w_{v_i, v_j}$ is the weight of the link between $v_i$ and $v_j$.

\subsubsection{Semantic Network}
\cite{bahramian_abbaspour_claramunt_2017} extend the information available for a tourism ontology so it can be used with spreading activation algorithm. A \textit{preference} and a \textit{confidence} values are associated to each ontology class. Each link has a weight that represents the degree of relationship between two classes or concepts. An ontology of different context scenarios or \textit{context factors} is linked to the tourism ontology, where the context factors are related to distance to POIs, time and weather information. The activation values of the nodes of the context ontology represent the level of fulfillment based on some measurement. The extended ontology is called \textbf{semantic network}.

\subsubsection{Learning Phase}
\cite{bahramian_abbaspour_claramunt_2017} defined the \textit{preference} and the \textit{confidence} for an ontology class $c$ during learning phase as follows:

\begin{equation} \label{eq:preference}
    pref_c = \frac{\displaystyle \sum_{p \in ancestors(c)}{conf_p pref_p}}
                    {\displaystyle  \sum_{p \in ancestors(c)} {conf_p}}
\end{equation}

\begin{equation} \label{eq:confidence}
    conf_c = \frac{\displaystyle \sum_{p \in ancestors(c)} {conf_p}}{|ancestors(c)|} - \alpha
\end{equation}
where $ancestors(c)$ is the set of ancestors of the ontology class $c$, $pref_c$ is the preference of the class $c$, $conf_c$ is the confidence of the class $c$ and $\alpha$ is the \textit{decrease rate} that will tell how much should decrease the \textit{confidence} at each level. These formulas are applied to each node traversing from the root or sources.

\subsubsection{Recommendation Phase}
To contextualize the recommendation, "\textit{the context factors are used as initial nodes in the spreading activation and transmit the activation flow}" in the work of \cite{bahramian_abbaspour_claramunt_2017}.

\subsection{Ontologies for e-tourism algorithm}

\section{Our Proposal}
\label{sec:proposal}

In this section we describe the architecture and functioning of our proposed User-Driven and Context-aware Hybrid Recommender System. Figure \ref{fig:arquitecture} shows its components:
\begin{itemize}
    \item \textbf{Data Gathering Module}: User preferences are received by the system, explicitly (direct interaction) or implicitly (data mining, social network analysis, etc.).
    \item \textbf{User Interest Module}: The preferences are propagated from higher classes to lower classes.
    \item \textbf{Context Module}: The system receives information about the context of the user, explicitly or implicitly (retrieved from an API, mobile information, etc.).
    \item \textbf{Recommendation Module}: The system recommends a set of places to the user.
\end{itemize}

In the following sections, we describe in detail each module.

\begin{figure}[h]
\centering
\includegraphics[scale=0.4]{draws/arquitecture.jpg}
\caption{System arquitecture}
\label{fig:arquitecture}
\end{figure}

\subsection{Data Gathering Module}
Initially, the system should receive initial \textbf{preferences}, as real values between $0$ and $1$, for the higher classes. To obtain these values, the user could explicitly set them by interacting with the system or they could be implicitly determined using data mining, social network analysis, geo-data, etc. During the lifetime of the system, the user can feed new updated preferences to the system.

\subsection{User Interest Model Module}
Inspired on the work of Bahramian et al. \cite{bahramian_abbaspour_claramunt_2017}, we introduce the concept of \textbf{semantic network}: a tourism ontology extended with the user's preferences (see figure \ref{fig:initial_pref}) and context factor links, a concept we explain later (see figure \ref{fig:init_act}). Just like Bahramian et al. \cite{bahramian_abbaspour_claramunt_2017}, we take advantage of the hierarchical shape of the semantic network for propagating the preference of superclasses to subclasses. Alongside preferences, each node of the semantic network has a \textbf{confidence} related to the user, a real value between $0$ and $1$ that defines how sure is the system that the computed preference is the real one. For computing the preference and the confidence we use formula \ref{eq:preference} and formula \ref{eq:confidence}, respectively:
\begin{equation} \label{eq:preference}
    pref_c = \frac{\displaystyle \sum_{p \in ancestors(c)}{conf_p pref_p}}
    {\displaystyle  \sum_{p \in ancestors(c)} {conf_p}}
\end{equation}
\begin{equation} \label{eq:confidence}
    conf_c = \frac{\displaystyle \sum_{p \in ancestors(c)} {conf_p}}{|ancestors(c)|} - \alpha
\end{equation}
where $ancestors(c)$ is the set of ancestors of the ontology class $c$, $pref_c$ is the preference of the class $c$, $conf_c$ is the confidence of the class $c$ and $\alpha$ is the \textit{decrease rate} that will tell how much should decrease the \textit{confidence} at each level. 

These formulas are applied to each node traversing from the higher classes, whose preferences are obtained from the Data Gathering Module, to the leaves. This process is called \textbf{preference propagation} and figures \ref{fig:initial_pref} and \ref{fig:pref_prop} show an example.

\begin{figure}[h]
\centering
\includegraphics[scale=0.5]{draws/initial_pref.jpg}
\caption{Initial preferences for "Cultural", "Store" and "Sport" classes}
\label{fig:initial_pref}
\end{figure}

\begin{figure}[h]
\centering
\includegraphics[scale=0.5]{draws/pref_spred.jpg}
\caption{Preference propagation for "Museum", "Art Gallery", "Souvenirs", "Stadium" and "Golf" classes, with decrease rate equal to 0.1}
\label{fig:pref_prop}
\end{figure}

\subsection{Context Module}
We define the \textbf{activation} of a node as the value that determines how feasible it is to go to a place that belongs to the node's ontology class in the current user's context. The user's context is determined by the \textbf{context factors}, entities that describe the characteristics of the user’s context that could affect their decision to go to a specific place. The context factors could be time, day and weather, as proposed by Bahramian \cite{bahramian_abbaspour_claramunt_2017}, but many others could be used, like transportation\cite{rajaonarivo2019rec} and mood. These entities are linked to the higher classes.

Let's define $f_x$ the \textit{fulfillment} of a context factor $x$ that has the value $1$ if $x$ fulfills or $0$ otherwise. Let's also define $r_{c,x}$ as the \textit{relevance} of context factor $x$ for node $c$, which is a real value in $[0, 2]$ that specifies how much the context factor can affect the decision to go to a POI that belongs to the node's ontology class; a value of $1$ means indifference; a value near $2$ means the fulfillment increases the wish to go to the POI; a value near $0$ means the fulfillment decreases the wish to go the POI. Hence, we define the activation for a node whose class $c$ is directly linked to the context factors:
\begin{equation} \label{eq:high_activation}
    act_c = \sum_{x \in contextFactors} r_{c,x} f_x
\end{equation}
and the activation for the internal nodes is defined as follows:
\begin{equation} \label{eq:activation}
    act_c = \frac{\displaystyle \sum_{c' \in ancestors(c)} act_{c'}}{|ancestors(c)|}
\end{equation}.
These formulas are used for an \textbf{activation propagation} from the higher classes. Figures \ref{fig:init_act}, \ref{fig:high_act} and \ref{fig:spread_act} show an example.

\begin{figure}[h]
\centering
\includegraphics[scale=0.45]{draws/initial_act.jpg}
\caption{System receives (sunny, night, weekend) as user context}
\label{fig:init_act}
\end{figure}

\begin{figure}[h]
\centering
\includegraphics[scale=0.45]{draws/high_act.jpg}
\caption{Compute activation for "Cultural", "Store" and "Sport" classes}
\label{fig:high_act}
\end{figure}

\begin{figure}[h]
\centering
\includegraphics[scale=0.45]{draws/spread_act.jpg}
\caption{Spread activation to "Museum", "Art Gallery", "Souvenirs", "Stadium" and "Golf" classes}
\label{fig:spread_act}
\end{figure}

Furthermore, this module receives the user's \textbf{location}, which will be used for querying near POIs.

\subsection{Recommendation Module}

We will first give two necessary definitions to compute the final score of an item and then we give the formula.

\subsubsection{Aging System}

Let's define $\eta_p$ as the POI $p$'s aging, initialized on $\eta_p = 1$. Let's define $H$ as the \textit{aging rate}. Each time a POI $p$ is recommended to the user, $\eta_p$ decreases by $H$. When $\eta_p < 0.1$, $\eta_p$ is reset to $1$. 

\subsubsection{Great-Circle distance}
Since the euclidean distance between two points on Earth would cross through the surface, we should use a more convenient measurement of distance: the \textit{great-circle distance} or \textit{orthodromic distance}. It is the shortest distance, along the surface of a sphere, between two points on the surface of the sphere. It is measured with circles on the sphere whose centers coincide with the center of the sphere. Those circles are called \textit{great-circles}. If we assume Earth is a perfect sphere and hence use Great-Circle distance, we get distances with errors no more than $0.5\%$ according to \cite{1997admiralty}. 

The distance between two points $i$ and $j$ on a sphere of radius $r$ is computed with the following formula:
\begin{equation} \label{eq:gc-dist}
    \begin{split}
        \scriptstyle{dist_{i,j} \ = \ r \cdot arccos (} & \scriptstyle{cos(lat_i) \cdot cos(lat_j) \cdot cos(lon_i - lon_j)} \\
                                        & \scriptstyle{+ \ sin(lat_i) \cdot sin(lat_j) )}
    \end{split}
\end{equation}

\subsubsection{Score} \label{section:score}
Let each $k_i$ be a parameter to the system, $dist_{u,p}$ be the great-circle distance between user $u$ and POI $p$ and $c$ be the node whose class is the one to which $p$ belongs. We define the \textit{score} of $p$ as a function that receives the maximum great-circle distance that a $p$ should be from $u$ as follows:
\begin{equation} \label{eq:score}
    \begin{split}
        score_p(maxdist) = \ &k_1 \cdot pref_c + k_2 \cdot act_c \\
                                        &+ k_3 \cdot \eta_p - k_4 \cdot \frac{dist_{u,p}}{maxdist}
    \end{split}
\end{equation}
\section{Study case: experiments and evaluation}
To evaluate our system we need some initial data to define the \textit{relevances} (see section \ref{section:context-awareness}) and some (real or simulated) users to test with.

\subsection{Relevances}
To define each $r_{c,x}$ for each node $c$ and context factor $x$, it was decided to gather real data provided by real people through a survey and then average them. The survey consisted of the question "Answer how much a 'context factor' influences your decision to go to a place/event of the specified type (these types are not exclusive)". Then, for each of the ontology classes specified on section \ref{section:context_factors} people had to answer a real value between $0$ and $10$, where:
\begin{itemize}
    \item $0$ means that if the context factor is met, you would not go to the place / event.
    \item 5 means you don't care if the context is met or not.
    \item 10 means that if the context is met, you would go to the place / event.
\end{itemize}

Then the following formula is applied to each feature or column, hence average them and transforming them to be on range $[0,2]$:
$$AVG(column)/5$$

The form had a total of 34 answers. Something worth to mention was that the answers were concentrated between the options $0$, $5$ and $10$, maybe because it is easier to think something between the thoughts "I would not", "I don't care" and "I would", than to think something that is in the middle of two of those thoughts.

\subsection{Users data}
As explained before (see section \ref{section:preferences-propagation}), a user of the system have to give initial preferences to some ontology classes. It was decided to make a form where people could answer their preferences of the higher level ontology classes mentioned before (see section \ref{section:context_factors}). The form had a total of 63 answers. The genre, country, profession, age and (optional) social networks were also asked to have more information for further work. The answer should be integers on range $[1, 10]$. Again, answers were more concentrated on the middle and highest options, but there were few low answers.

\subsubsection{Simulated scenarios}
To evaluate the system, it was decided to test it with simulated scenarios (simulated context and simulated user). A set of clusters were chosen as the set of simulated users. The \textit{K-Means} algorithm was used to get the clusters and the \textit{Elbow Method} (see figure \ref{fig:elbow}) was used to chose $4$ as the number of clusters.
\begin{figure}[h]
    \centering
    \includegraphics[scale=0.45]{elbow.png}
    \caption{Distortions of the different number of clusters}
    \label{fig:elbow}
\end{figure}

\subsection{Experiments}
For each centroid of the clusters mentioned previously, we simulate two visits to each possible scenario using each possible value for each context factor. The system returns a set of not more than $5$ recommended places inside a radius of $8$ kilometers for each visit. For each recommended set we analyze the average preference, average activation, average aging, average novelty \cite{kotkov2016survey} and average \textit{distance}. To compute novelty we use the metric mentioned on section \ref{section:serendipity}, which is adapted as follow:
\begin{equation} \label{eq:novelty}
    nov(p, u) = \  \underaccent{q \in rec(u)}{min} \  ( dist( c_p, c_q ) )
\end{equation}
where $p$ is a recommended place, $u$ the user to which $p$ was recommended, $rec(u)$ is the set of already recommended places to $u$, $c_p$ is the ontology class to which $p$ belongs and $dist$ computes the distance in the ontology (using \textit{Breadth First Search}) of two ontology classes.
\section{Conclusions and future work} \label{section:conclu}

In this paper we present our proposal for a context-aware recommender system that is able to recommend varied Points of Interest (POI), hence increasing serendipity. In order to do so, we use a spreading activation algorithm for the user model and the contextual model and we use a proposed aging system which gives more priority to POIs that are not frequently recommended. This system should be integrated with a mobile app or a web app able to gather user information and contextual information, including location.

After describing our proposal, we explain some experiments for testing our system. These experiments are done simulating interactions of some simulated tourists, based on real preferences obtained through a survey. In future work, this system should be deployed and be tested by real interactions and receive feedback about the relevance, suitability regarding context and serendipity of the recommendations.

For measuring novelty we use the Eq. (\ref{eq:novelty}) based on Kotkov et al. \cite{kotkov2016survey}, since when novelty increases, so does serendipity. That equation was proposed with the hypotheses that if a user already knows places from class $c$, then it is probable the user already knows other places from class $c$. With that hypotheses, our system gives poor novelty: when a place belonging to the ontology class $c$ is recommended, next recommendations with places from $c$ will have lower novelty. However, considering average aging as a good metric for the level of variability, both experiments on section \ref{section:experiments} give high average aging, which means a lot of "young" or not frequently recommended places are returned to the user, which could imply an increment on system serendipity. Future work should try more advanced ways for measuring serendipity.

Of course, experiment \ref{section:experiment-2} is focused on increasing average aging by giving a higher value to $k_3$ on Eq. (\ref{eq:score}), but the increment of aging is paid with a decrement on activation. Nonetheless, two simulated tourists have an increment for their preferences, because on first configuration the system is trapped between very activated POIs but not very relevant, but the heavier aging of the second configuration increases variability, hence lets the system recommend other POIs despite of lowering the activation. 

Nevertheless, a deployed version of this system should let the users choose each $k_i$ of Eq. (\ref{eq:score}), that way choosing if they want recommendations mainly relevant than contextually activated or varied, or even recommendations with the nearest places ignoring any other attribute, or any other possible configuration.


%\medskip
 
 \bibliographystyle{unsrt}
 \balance
\bibliography{bibliography}
%\printbibliography

\end{document}
