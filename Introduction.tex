\section{Introduction}
\textcolor{red}{XXXLa intro y el abstract deben tener la siguiente estructura: hablar del contexto general: Turismo y e-tourismo dado los avances tecnológicos; lo que se usa y sus limitaciones (que es el párrafo que tienes ahora); luego lo que se propone (la contribución del paper, la conclusión general de los resultados y finalmente la organización del paperXXXX}

\textcolor{red}{XXX Pondré en rojo XXXXMis comentariosXXXXXXXX y los párrafos que redacte/cambie. Así los ubicas más fácilmente}


\textcolor{red}{Tourism is one of the most promising areas for diverse applications and is becoming an extremely
important market\cite{buhalis2011tourism,murphy2013tourism,fermoso2015open,ku2015cultivating,alghamdi2016tourism,artemenko2017tourism,kazandzhieva2019tourism,jannach2020interactive}. 
When travellers are going or planning to go to a destination, they try to collect information about their new destination “as best they can” (e.g., by going to the tourist office or by obtaining information about their destination and its surroundings on the Internet). Usually, tourists depend on places' reviews to decide where to spend their vacations, their free time, or even their lunch or just their afternoon, but this requires prior knowledge of the existence of the places. Furthermore, sometimes people know about a place but do not know if it is a good fit for their necessities and ignores it, taking the risk of missing a good experience. Nevertheless, this requires a considerable amount of effort on the part of the tourist. }


%\textcolor{red}{Based on the observation that tourism activity is strongly correlated with travellers’ preferences and interests,
%there is a need for applications that automatically extract tourist information from the data shared, accessed, and managed by tourists and thus offer content that is more relevant and tailored to a given user’s profile (e.g., photos, images, social networks, friends’ preferences, and sentiments).
%With the rapid evolution of mobile devices and social media platforms, more and more users are connected and collaborate by sharing information via social networks such as Facebook or Instagram. These mobile devices and social platforms offer users the ability to share and manage large amounts of comments and data, as well as a large amount of multimedia objects (photos,
%videos) produced by users during their daily activities (travel, tourism, events). The data produced by the users reflect their personal tastes and interests (favourite destinations, artistic or culinary tastes, types of stays, etc.), and allow the inference of information to make suggestions or
%recommendations to the users in a particular context (e.g., by natural language processing and sentiment analysis in text and images). XXXREFERENCIAS XXX}


%Usually, people depend on places' reviews to decide where to spend their vacations, their free time or even their lunch or just their afternoon, but this requires prior knowledge of the existence of the places. Furthermore, sometimes people know about a place but do not know if it is a good fit for their necessities and ignores it, taking the risk of missing a good experience.

\textcolor{red}{Currently, there is a plethora of applications offering information on cities and historical centres, points of interest (POIs) to visit, city tours, green areas to rest, etc. The use of such applications implies that the user is first aware of them, then installs them in the hope that they are suitable (only 10\% of downloaded mobile applications are used more than once), learns and knows how
to use them and then filters the data/information that is uploaded or offered. Most of these
applications follow the PULL paradigm, i.e., the user has to explicitly search for
suggestions/recommendations through interaction with the application’s graphical user interface.
We believe that this is one of the reasons why these types of applications are not successful: they
require too many steps, too many interactions, and too many unknowns. On the other hand, a
PUSH approach, in which the application proactively triggers a recommendation process when
necessary (for example, when the user is close to a POI, when lunch time is approaching, or
when weather conditions change) seems to be a more reasonable solution. A first step in this
direction has been taken by Google via Google Now\footnote{https://www.google.com/intl/fr/landing/now/}, 
which provides information/suggestions
to users by detecting in their geographical environment what they need according to their
location and time. However, such applications are simplistic and use only a limited amount of
information, and moreover, do not take into account the user’s activity. A key element is the
consideration of the users' context, such as their locations, weather (\textit{is it raining?}), or time (\textit{is it night?}).}


\textcolor{red}{Recommender Systems are now at the heart of
much research and offer real benefits to users, organizations, and the business community in
general~\cite{borras2014intelligent,del2016pull,leskovec2020mining}. In the context of tourism, most recommender systems are able to predict user's preferences based on previous activity, but research on Context-Aware Recommender Systems
(CARS) are still to
be further developed~\cite{adomavicius2011context,nejma2015service,haruna2017context,raza2019progress}. There is a real need for research and design of innovative solutions in this field.
}



%This is why a recommender system is useful for tourism, a system able to predict user's preferences based on previous activity, however this kind of systems do not usually consider user's context, such as weather (\textit{is it raining?}) or time (\textit{is it night?}). This motivates us to build a recommender system for tourism aware of user's context, able to recommend a set of places of interest that will meet his preferences, based on his previous activity.



\textcolor{red}{In this context, in this work we propose XXXXX In this sense, the envisaged
recommendation system will have to allow the user to choose a recommendation level to have a
greater probability of making happy discoveries in order to sharpen their curiosity. Also
according to the users’ wishes, the serendipity of recommendations may increase (e.g., surprise
events, unrepeated recommendations)~\cite{kotkov2016survey}. It is also desirable that the user knows why
something is recommended (i.e., that the recommendations are explainable) XXX ACOMODARXXXXXXX.
}

\textcolor{red}{We organize the rest of this paper as follows. In Section XXXXCOMPLETAR AL FINAL XXXXX}


