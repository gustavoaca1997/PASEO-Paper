\section{Introduction}
\textcolor{red}{XXXLa intro y el abstract deben tener la siguiente estructura: hablar del contexto general: Turismo y e-tourismo dado los avances tecnológicos; lo que se usa y sus limitaciones (que es el párrafo que tienes ahora); luego lo que se propone (la contribución del paper, la conclusión general de los resultados y finalmente la organización del paperXXXX}

\textcolor{red}{XXX Pondré en rojo XXXXMis comentariosXXXXXXXX y los párrafos que redacte/cambie. Así los ubicas más fácilmente}


\textcolor{red}{Tourism is one of the most promising areas for mobile applications and is becoming an extremely
important market\cite{buhalis2011tourism,murphy2013tourism,fermoso2015open,ku2015cultivating,alghamdi2016tourism,artemenko2017tourism}. 
When travellers are going or planning to go to a destination, they try to
collect information about their new destination “as best they can” (e.g., by going to the tourist
office or by obtaining information about their destination and its surroundings on the Internet).
Nevertheless, this requires a considerable amount of effort on the part of the tourist. Based on the
observation that tourism activity is strongly correlated with travellers’ preferences and interests,
there is a need for applications that automatically extract tourist information from the data shared,
accessed and managed by tourists and thus offer content that is more relevant and tailored to a
given user’s profile (e.g., photos, images, social networks, friends’ preferences and sentiments).
With the rapid evolution of mobile devices and social media platforms, more and more users are
connected and collaborate by sharing information via social networks such as Facebook or
Instagram. These mobile devices and social platforms offer users the ability to share and manage
large amounts of comments and data, as well as a large amount of multimedia objects (photos,
videos) produced by users during their daily activities (travel, tourism, events). The data produced
by the users reflect their personal tastes and interests (favourite destinations, artistic or culinary
tastes, types of stays, etc.), and allow the inference of information to make suggestions or
recommendations to the users in a particular context (e.g., by natural language processing and
sentiment analysis in text and images). XXXXACOMODARXXXX}



Usually, people depend on places' reviews to decide where to spend their vacations, their free time or even their lunch or just their afternoon, but this requires prior knowledge of the existence of the places. Furthermore, sometimes people know about a place but do not know if it is a good fit for their necessities and ignores it, taking the risk of missing a good experience. This is why a recommender system is useful for tourism, a system able to predict user's preferences based on previous activity, however this kind of systems do not usually consider user's context, such as weather (\textit{is it raining?}) or time (\textit{is it night?}). This motivates us to build a recommender system for tourism aware of user's context, able to recommend a set of places of interest that will meet his preferences, based on his previous activity.
