\vspace{-0.2cm}
\section{Introduction}
%\textcolor{red}{XXXLa intro y el abstract deben tener la siguiente estructura: hablar del contexto general: Turismo y e-tourismo dado los avances tecnológicos; lo que se usa y sus limitaciones (que es el párrafo que tienes ahora); luego lo que se propone (la contribución del paper, la conclusión general de los resultados y finalmente la organización del paperXXXX}

%\textcolor{red}{XXX Pondré en rojo XXXXMis comentariosXXXXXXXX y los párrafos que redacte/cambie. Así los ubicas más fácilmente}

%Lets imagine a person in Paris, fan of museums and parks. That person wishes to spend the afternoon in a good place but it is a rainy day, so the experience should not be affected by the bad weather. That person would like to use a service or app that recommends a nearby good place for spending this rainy afternoon, preferably a museum instead of a park, hence spending no time searching for possible places to visit.

Tourism is one of the most promising areas for diverse applications and is becoming an extremely
important market~\cite{buhalis2011tourism,murphy2013tourism,fermoso2015open,ku2015cultivating,alghamdi2016tourism,artemenko2017tourism,kazandzhieva2019tourism}. 
When travellers are going or planning to go to a destination, they try to collect information about their new destination “as best as they can” (e.g., by going to the tourist office or by obtaining information about their destination and its surroundings on 
%the 
Internet). Usually, tourists depend on places' reviews to decide where to spend their vacations, their free time, or even their lunch or just their afternoon, but this requires prior knowledge of the existence of the places. Furthermore, sometimes people know about a place but do not know if it is a good fit for their necessities and ignores it, taking the risk of missing a good experience. Nevertheless, this requires a considerable amount of effort on the part of the tourists.

Currently, there is a plethora of applications offering information on cities and historical centres, points of interest (POI) to visit, city tours, green areas to rest, etc. The use of such applications implies that the user is first aware of them, then installs them in the hope that they are suitable (only 10\% of downloaded mobile applications are used more than once), learns and knows how
to use them and then filters the 
%data/
information.
%that is uploaded or offered. 
Most of these
applications follow the PULL paradigm, i.e., the user has to explicitly search for
suggestions/recommendations through interaction with the application’s graphical user interface.
We believe that this is one of the reasons why these types of applications are not successful: they
require too many steps, too many interactions, and too many unknowns. On the other hand, a
PUSH approach, in which the application proactively triggers a recommendation process when
necessary (for example, when the user is close to a POI, when lunch time is approaching, or
when weather conditions change) seems to be a more reasonable solution. A first step in this
direction has been taken by Google via Google Now\footnote{https://www.google.com/intl/fr/landing/now/}, 
which provides information/suggestions
to users by detecting in their geographical environment what they need according to their
location and time. However, such applications are simplistic and use only a limited amount of
information, and moreover, do not take into account the user’s activity. A key element is the
consideration of the users' context, such as their locations, weather (\textit{is it raining?}), or time (\textit{is it night?}).

Recommender systems are now at the heart of
much research and offer real benefits to users, organizations, and the business community
%in general
~\cite{borras2014intelligent,del2016pull,lim2019tour,leskovec2020mining}.
%In the context of tourism, 
Most tourism recommender systems are able to predict user's preferences based on previous activities, but research on Context-Aware Recommender Systems
%(CARS) 
are still to be further developed~\cite{adomavicius2011context,haruna2017context,raza2019progress}. 
Additionally, most of them
%recommender systems 
do not consider to vary recommendations. %-- i.e., they become predictable  for same users, under the same situations. 
Thus, there is still a real need for research and design of innovative solutions in this field. Nowadays, the use of ontologies to represent the knowledge related to tourism (POI, users' information, context information, etc.) is becoming a powerful tool to offer such as innovative solutions~\cite{borras2014intelligent,yochum2020linked}.
%\cite{rajaonarivo2019rec} \cite{bahramian_abbaspour_claramunt_2017}, 
%being DATAtourisme\footnote{http://info.datatourisme.gouv.fr/ontology/core/2.0/} a good example.

Imagine a tourist in Paris, fan of museums and parks, who wishes to spend the afternoon in a good place, but it is a rainy day. The experience should not be affected by the bad weather. 
%This  tourist 
She/He would like to use 
%a service 
%or 
an app 
that recommends a nearby good place for spending this rainy afternoon, preferably a museum instead of a park, hence spending no time searching for possible places to visit. Moreover, if in the next day it is still raining (it rains a lot in Paris!), 
%(il pleut beaucoup à Paris !), 
the tourist would like to visit a different museum or other %touristic 
indoor place.

%In this context, 
In this work we propose RECESO, a {\bf Rec}ommender system for {\bf E}-tourism with {\bf S}erendipity and {\bf O}ntology-based. RECESO is
%a 
user-centric 
%hybrid 
%recommender system that 
and supports both PULL and PUSH approaches. We propose 
%, using 
an ontology-based spreading activation algorithm for %content-based and 
context-aware recommendations. According to the users’ wishes, the serendipity of recommendations may increase (e.g., surprise events, no repeated recommendations)~\cite{kotkov2016survey},
%. All this 
based on an {\it aging-like} algorithm, which gives less priority to recently recommended places.  It is also desirable that the user knows why something is recommended (i.e., 
%that 
the recommendations are explainable). We describe the architecture and algorithms of RECESO
%our recommender system. 
%To 
and demonstrate its suitability and performance, 
%we test it 
in different simulated scenarios, with different users' profiles and preferences and different context parameters.
%using different weights or priorities for the predicted preferences, context activation, aging, and distances of the recommended places. 
Results show how the spreading activation and the proposed aging system 
give 
relevant and varied recommendations, according to users' preferences,
%for the users, 
while taking into account their context. We also
%, and 
show how the system could be configured to focus more on either relevant, varied, context-activated, or near places. 

%We organize the rest of this paper as follows. In Section \ref{section:related-work}, we survey 
%%introduces 
%relevant studies related to our work.
%%and concepts for understanding our system. 
%Section \ref{sec:proposal} describes the architecture and functioning of our proposed system. Section \ref{section:study-case} describes results obtained during the experiments. Finally, Section \ref{section:conclu} explains our final thoughts.


