\section{Conclusions and future work} \label{section:conclu}

In this paper we present our proposal for a context-aware recommender system that is able to recommend varied Points of Interest (POI), hence increasing serendipity. In order to do so, we use a spreading activation algorithm for the user model and the contextual model and we use a proposed aging system which gives more priority to POIs that are not frequently recommended. This system should be integrated with a mobile app or a web app able to gather user information and contextual information, including location.

After describing our proposal, we explain some experiments for testing our system. These experiments are done simulating interactions of some simulated tourists, based on real preferences obtained through a survey. In future work, this system should be deployed and be tested by real interactions and receive feedback about the relevance, suitability regarding context and serendipity of the recommendations.

For measuring novelty we use the equation \ref{eq:novelty} based on Kotkov et al. \cite{kotkov2016survey}, since when novelty increases, so does serendipity. That equation was proposed with the hypotheses that if a user already knows places from class $c$, then it is probable the user already knows other places from class $c$. With that hypotheses, our system gives poor novelty: when a place belonging to the ontology class $c$ is recommended, next recommendations with places from $c$ will have lower novelty. However, considering average aging as a good metric for the level of variability, both experiments on section \ref{section:experiments} give high average aging, which means a lot of "young" or not frequently recommended places are returned to the user, which could imply an increment on system serendipity. Future work should try more advanced ways for measuring serendipity.

Of course, experiment \ref{section:experiment-2} is focused on increasing average aging by giving a higher value to $k_3$ on equation \ref{eq:score}, but the increment of aging is paid with a decrement on activation. Nonetheless, two simulated tourists have an increment for their preferences, because on first configuration the system is trapped between very activated POIs but not very relevant, but the heavier aging of the second configuration increases variability, hence lets the system recommend other POIs despite of lowering the activation. 

Nevertheless, a deployed version of this system should let the users choose each $k_i$ of equation \ref{eq:score}, that way choosing if they want recommendations mainly relevant than contextually activated or varied, or even recommendations with the nearest places ignoring any other attribute, or any other possible configuration.