\section{Related Work} \label{section:related-work}

% \textcolor{red}{Clasificar los sistemas de recomedación de acuerdo a: User-centric (manual, automatic); context-aware; ontology-based; used methods (ML, DL, heurísticas, modelos, etc.)\\
% INCLUIR (y otros):\\

% - Efficient user profiling based intelligent travel recommender system for individual and group of users, 2019 NO PDF\\
% - Ingrid Christensen, Silvia Schiaffino, and Marcelo Armentano.Social group recommendation in the tourism domain. Journal of intelligent information systems, 47(2):209–231, 2016\\ NO PDF
% - L Rizaldy Hafid Arigi, Z K Abdurahman Baizal, and Anisa Herdiani. Context-aware recommender system based on ontology for recommending tourist destinations at bandung. Journal ofPhysics: Conference Series, 971:012024, mar 2018. } NO PDF

%\subsection{Recommender Systems} %\label{section:recommender-systems}

The unleashed proliferation  of recommender systems for tourism domain, due to the current development of mobile Internet technology, has fostered a wide categorization of them. 
Depending on the technique used to resolve recommendations, these systems are usually classified into two categories: memory-based and model-based~\cite{bobadilla2013recommender,ebrahim_2012}. \textbf{Memory-based} recommender systems support their  suggestions on similarities among users and their shared items, hence recommending similar items to the ones the user likes (i.e., \textit{content-based recommendation}) or recommending items liked by users that are similar to the user (i.e., \textit{collaborative filtering}). \textbf{Model-based} recommender systems try to guess how much a user will like an item that they did not consume before, usually through statistical techniques and machine-learning models. The recommender systems that use both memory-based and model-based approaches are called \textbf{Hybrid} recommender systems.

In this section, we survey some relevant and recent studies in the tourism domain, classifying them according to these three categories and considering aspects related to  how they manage users' preferences and interests, the context awareness, the use of ontologies, and the variability of recommendations. We compare the related work in terms of these criteria and highlight the difference with our proposal.    

\subsection{User information}
Obviously, all recommender systems take into account some information related to the user, such as identification aspects (sex, age, profession, etc.), interest and preferences, or social relations. This information can be obtained from explicit (e.g., questionnaires)~\cite{jannach2020interactive} or implicit (e.g., by analyzing feedbacks in user's social networks) media~\cite{lin2018hybrid}.  

Some proposals are supported on users' interaction to obtain their data (i.e., following a PULL paradigm). In  %Rajaonarivo et al.
~\cite{rajaonarivo2019rec}, authors model the users' information considering their gender, age category (e.g., kid, adult, or elderly), and preferences, classified as thematic preferences (e.g., museum, theater) and historical preferences (e.g., 12th century), that have to be provided by them through a user interface.  In~\cite{santos2019using}, only health,  physical, and psychological conditions are required  with forms filled by users. 
Users' preferences and feedback are directly assigned by the users in the proposal presented in~\cite{bahramian_abbaspour_claramunt_2017}. As in ~\cite{arigi2018context}, users' interest (i.e., not interested, less interested, and interested enoughd) on tourism categories are asked to them. SMARTMUSEUM~\cite{ruotsalo2013smartmuseum}, asks users about the desired duration of a visit to a particular location, the motivation for a visit, and ability to consume the content offered by the system. The system proposed in~\cite{shen2016attraction} requires photos uploaded by users, from which it extracts their travel history; it also asks users to rank  POI (their favorite and non-favorite attractions).
 

Other recommender systems extract users' information, mainly their preferences and interests, from available sources,
without asking for explicit interaction.  The work presented in~\cite{kesorn2017personalized}, takes from Facebook basic information (e.g., name, age) and  check-in data (e.g., visited places)  to identify users' interests and preferences. 
To automatically deduce users' preferences from their social networks, many techniques are used, such as opinion mining~\cite{logesh2019exploring,logesh2018personalised}, analysis of implicit/explicit feedback~\cite{hidasi2016general}, and analyzing geotagged pictures~\cite{sun2019building}. 
Curumim~\cite{menk2017curumim} takes from users' social networks their travel history and level of education, and predicts their degree of curiosity. Most of these works follow the PULL approach.


Many other proposals combine both explicit and implicit users' information gathering, also under the PULL paradigm. SPETA~\cite{garcia2009speta},  supports  its recommendation on the interest and rating  of tourism places that users explicitly provide, and on preferences deduced by analyzing their behavior on social networks. The system presented in~\cite{alonso2012ontology}, takes into account special needs and  context-dependent preferences on tourism sites, directly  provided by users, as well as explicit/implicit feedback on their social networks. 

\subsection{Context-awareness}
In order to improve suggestions, the trend is to consider context aspects that describe a specif situation in a determined moment for a user, including transportation media, weather, time, or even health conditions. 

Many works use context-modeling approaches that mainly consider means of transportation, travel time,  location, or weather~\cite{rajaonarivo2019rec}\cite{bahramian_abbaspour_claramunt_2017}\cite{arigi2018context}\cite{kesorn2017personalized}\cite{logesh2019exploring} \cite{logesh2018personalised}.


SMARTMUSEUM~\cite{ruotsalo2013smartmuseum} only considers users' location as contextual information captured by the built-in sensors of mobile devices, such as GPS, accelerometers, and RFID readers. The system proposed in~\cite{shen2016attraction} collects automatically the users' current location (city, latitude, and longitude) and current time, thus recommendations are influenced by the geo-distance to POI. SPETA~\cite{garcia2009speta} also considers users' location extracted from users' mobile devices, but it takes into account current and the history of past locations. It also gathers from  mobile devices other contextual information, such as weather forecast and time. 


A General Factorization Framework for context-aware recommender systems is proposed in~\cite{hidasi2016general}, with the aim of constructing factorization matrices (not matter which and how many context factors are considered) for machine learning techniques. In \cite{sun2019building}, it is used a combination of contextual information like weather, transportation, or textual information, with geotagged pictures taken by the user for building the user model.


%The work proposed in~\cite{rajaonarivo2019rec},
%Rajaonarivo et al.
% uses a
%context-modeling approach that considers means of transportation, travel time, and location. LISTO

%Kesorn et al. \cite{kesorn2017personalized} use the time of the day for achieving more suitable recommendations. LISTO

The system proposed in~\cite{alonso2012ontology}, is the only one among the referenced works that considers context-dependent   preferences, as our proposal,  related to weather, day, transport, and special needs. It means that users' preferences for POI, are expressed with regards the context (e.g., {\it indoor places when is raining}, {\it art exhibitions at night}).


%Logest et al. \cite{logesh2019exploring} \cite{logesh2018personalised} also consider user context like user location.  LISRO


%Bahmarian et al. \cite{bahramian_abbaspour_claramunt_2017} consider weather, time, and day. LISTO


\subsection{Serendipity}

A useful goal for a recommender system is to be serendipitous, which 
%accordingly to Kotkov et al. \cite{kotkov2016survey} 
means that the recommended items must be relevant, novel, and unexpected. A relevant item is an item that the user likes, consumes, or is interested in; a novel item is an item that users have never seen or heard about in their life; an unexpected item is an item that significantly differ from the profile of the user. Some quite good recent surveys emphasize the importance of such as feature for recommender systems in general~\cite{kotkov2016survey,chen2019serendipity} and in the tourism domain~\cite{tintarev2010off,menk2019recommendation}.

%The same work explains the differences between novelty and unexpectedness: "First, a user may find an item unexpected even if she/he is familiar with the item. Second, to surprise a user, unexpected items have to differ from the user profile more than novel items".

It starts to be a trend to recommend  POI. The model considered in~\cite{rajaonarivo2019rec}, considers previous activity of other users to recommend tours of POI, thus reducing overspecialization and increasing serendipity, however experiments with this feature were not made. Authors of~\cite{shen2016attraction} assert that the proposed system is able to recommend fresh and surprise POI, based on collective intelligence.
 From the level of curiosity predicted from users' social networks, Curumim~\cite{menk2017curumim}  adapts the degree of surprise and unexpectedness of a recommended POI, tailored to users' curiosity values. 






%Rajaonarivo et al. \cite{rajaonarivo2019rec} model considers previous activity of other users to recommend tours of POI, thus reducing overspecialization and increasing serendipity, however experiments with this feature were not made.

\subsection{Use of ontologies}

The huge amount of data that can be managed in recommender systems, related to users' information, users' preferences, context factors, POI, etc.,   demands the use of more complex knowledge. Even though, recommender system is an area that has been the focus of many studies, thus reaching a very good level of maturity, there is still a lack of standardization to represent such information. In this sense, it is evident the necessity of a well-defined and standard model for representing the knowledge managed by recommender systems. Semantic Web, in particularly the use of ontologies, seems to be a clear solution, from which we can take its organizational and relational capacity.
In the context of tourism, ontology-based  recommender  system  is  an  emerging  trend~\cite{borras2014intelligent,yochum2020linked}.

Some systems only count on ontologies to represent tourism  POI~\cite{rajaonarivo2019rec,bahramian_abbaspour_claramunt_2017,garcia2009speta,arigi2018context}. Other works consider user profile ontologies, besides tourism ontologies, such in~\cite{ruotsalo2013smartmuseum}. In~\cite{alonso2012ontology}, an ontology to represent user preferences and context factors is proposed.



%Rajaonarivo et al. \cite{rajaonarivo2019rec} export the data used to well known open data platforms before using it; 

%Bahmarian et al. 
%\cite{bahramian_abbaspour_claramunt_2017} based their proposal on spreading activation algorithms on the nodes of an e-tourism ontology.


\subsection{Discussion}

Table \ref{table:related-work} shows a comparative evaluation  of the referenced works, in terms of \textbf{category} of the system (model-based, hybrid, etc.), type of \textbf{user's information} gathered, \textbf{context} factors considered, \textbf{ontology} used, and how they handle \textbf{serendipity}. 

Rajaonarivo et al. \cite{rajaonarivo2019rec} is the only work that proposes a context-aware system using ontologies and approaching serendipity, as our proposed system RECESO, nonetheless they did not experiment the serendipitous feature. In contrast, our work does evaluate the serendipity of our proposed system. Moreover, RECESO supports the PUSH paradigm by implicitly gathering users' preferences and context, and the PULL paradigm by explicit user interactions.

%Our work does evaluate the serendipity of our proposed system, which can support the PUSH paradigm by implicitly gathering users' preferences and context, and the PULL paradigm by explicit user interactions.

\begin{table*}[h!]
    \centering
    \caption{Related work on recommender systems for e-tourism}
    \label{table:related-work}
    \begin{tabular}{|c|c|c|c|c|c|} 
        \hline
        \textbf{Work} & \textbf{Category} & \textbf{Users' information} & \textbf{Context}&\textbf{Ontology}&\textbf{Serendipity} \\
        \hline \hline 

        \cite{rajaonarivo2019rec} & Hybrid & Basic, Preferences & Transportation, Location, Weather & Tourism & Collaborative \\ \hline

        \cite{santos2019using} & Model-based & Health & & & \\ \hline

        \cite{bahramian_abbaspour_claramunt_2017} & Hybrid & Preferences & Weather, Time, Day & Tourism, Context & \\ \hline

        \cite{arigi2018context} & Model-based & Preferences & Transportation, Location, Weather & Tourism & \\ \hline

        \cite{ruotsalo2013smartmuseum} & Model-based & Duration, Motivation, Ability & Location & Tourism, User & \\ \hline

        \cite{shen2016attraction} & Model-based & Preferences, travel history & Location, Time & & Surprise \\ \hline

        \cite{kesorn2017personalized} & Hybrid & Social, Check-in data & Transportation, Location, Weather & & \\ \hline

        \cite{logesh2019exploring} & Hybrid & Social, Opinion mining & Transportation, Location, Weather &  & \\ \hline

        \cite{logesh2018personalised} & Hybrid & Social, Opinion mining & Transportation, Location, Weather &  & \\ \hline

        \cite{hidasi2016general} & Model-based & Social, Feedback  & General &  & \\ \hline

        \cite{sun2019building} & Model-based & Social, Pictures & Transportation, Weather, Textual &  & \\ \hline

        \cite{menk2017curumim} & Hybrid & Social, History, Education &  &  & Curiosity \\ \hline

        \cite{garcia2009speta} & Hybrid & Preferences, Social & Location & Tourism & \\ \hline

        \cite{alonso2012ontology} & Hybrid & Social, Preferences & Weather, Day, Transport, Special Needs & Tourism, Context & \\ \hline

        \hline
        RECESO & Hybrid & Preferences & Weather, Time, Day, Location & Tourism & Aging \\ \hline

        \hline
    \end{tabular}
    
    \end{table*}


% \subsection{Usage of ontologies: Spreading Activation} \label{section:usage-spreading-activation}
% Previous work \cite{bahramian_abbaspour_claramunt_2017} integrates the concepts of recommender systems with ontologies at building a context-aware tourism recommender system based on the \textit{spreading activation} algorithm. In its authors' own words: "In spreading activation method, a given concept is represented by a node and has an activation value. A relation among different concepts is represented by a link between nodes and has a weight value. To initialize the algorithm, one or several nodes of a network are activated and these activations spread to the relevant nodes. This process is iterated until a stopping condition (e.g., number of node processed) is reached". The activation value of the node \(v_i\) is computed as follows:
% \begin{equation} \label{eq:og_activation}
% a_{v_i} = \sum_{v_j \in neighbors(v_i)} w_{v_i, v_j} a_{v_j} 
% \end{equation}
% where $a_{v_i}$ is the activation value of node $v_i$, $neighbors(v_i)$ is the set of $v_i$'s neighbor nodes and $w_{v_i, v_j}$ is the weight of the link between $v_i$ and $v_j$. This concept will be useful in further sections. 

% \subsubsection{Semantic Network}
% Behmarian et al. \cite{bahramian_abbaspour_claramunt_2017} extend the information available for a tourism ontology so it can be used with spreading activation algorithm. A \textit{preference} and a \textit{confidence} values are associated to each ontology class. Each link has a weight that represents the degree of relationship between two classes or concepts. An ontology of different context scenarios or \textit{context factors} is linked to the tourism ontology, where the context factors are related to distance to POIs, time and weather information. The activation values of the nodes of the context ontology represent the level of fulfillment based on some measurement. The extended ontology is called \textbf{semantic network}.

% \subsubsection{Learning Phase}
% Bahmarian et al. \cite{bahramian_abbaspour_claramunt_2017} defined the \textit{preference} and the \textit{confidence} for an ontology class $c$ during learning phase as follows:

% \begin{equation} \label{eq:preference}
%     pref_c = \frac{\displaystyle \sum_{p \in ancestors(c)}{conf_p pref_p}}
%                     {\displaystyle  \sum_{p \in ancestors(c)} {conf_p}}
% \end{equation}

% \begin{equation} \label{eq:confidence}
%     conf_c = \frac{\displaystyle \sum_{p \in ancestors(c)} {conf_p}}{|ancestors(c)|} - \alpha
% \end{equation}
% where $ancestors(c)$ is the set of ancestors of the ontology class $c$, $pref_c$ is the preference of the class $c$, $conf_c$ is the confidence of the class $c$ and $\alpha$ is the \textit{decrease rate} that will tell how much should decrease the \textit{confidence} at each level. These formulas are applied to each node traversing from the root or sources.

% \subsubsection{Recommendation Phase}
% To contextualize the recommendation, "\textit{the context factors are used as initial nodes in the spreading activation and transmit the activation flow}" \cite{bahramian_abbaspour_claramunt_2017}, then the activated nodes are recommended. 

% The feature of using sets of similar items without using statistical models such as clustering but using ontology classes, and also predicting the user preferences, makes this kind of systems to be hybrid recommender systems.

% \subsection{Serendipity} \label{section:serendipity}
% An useful goal for a recommender system is to be serendipitous, which accordingly to Kotkov et al. \cite{kotkov2016survey} means that the recommended items must be relevant, novel and unexpected. A relevant item is an item that the user likes, consumes or is interested in; a novel item is an item that the user has never seen or heard about in their life; an unexpected item is an item that significantly differ from the profile of the user. The same work explains the differences between novelty and unexpectedness: "First, a user may find an item unexpected even if she/he is familiar with the item. Second, to surprise a user, unexpected items have to differ from the user profile more than novel items.".

% Kotkov et al. \cite{kotkov2016survey} define many ways to measure how serendipitous a recommender system is, through measuring first relevance, novelty and unexpectedness of its recommendations. Since a recommender system will probably predict the preference a user has for a specific item, hence predict the relevance of the item, we are mainly focused on novelty and unexpectedness.

% There is a metric proposed by Nakatsuji et al. (as cited in \cite{kotkov2016survey}) for novelty that uses the taxonomy classes to which each recommended item belongs and their distance between them, being the minimum distance of a new recommended item $i$ to each already recommended item $j$ the novelty of the item $i$:
% \begin{equation}
%     nov_{na}(i, u) = \underaccent{j \in I_u}{min}(dis(cls_j, cls_i))
% \end{equation}
% where $cls_k$ is a class of item $k$ in the taxonomy and $dis(cls_j, cls_i)$ is the distance between classes $cls_j$ and $cls_i$ in the taxonomy. Since our work is focused on the use of an ontology on tourism field, this metric is very helpful.

% % \subsection{Ontologies for e-tourism algorithm}
