\section{Related Work}

\textcolor{red}{Clasificar los sistemas de recomedación de acuerdo a: User-centric (manual, automatic); context-aware; ontology-based; used methods (ML, DL, heurísticas, modelos, etc.)\\
INCLUIR (y otros):\\
- E-tourism recommender systems: a survey and development perspectives, 2017, \cite{artemenko2017tourism}\\
-Exploring hybrid recommender systems for personalized travel applications, 2019, \cite{logesh2019exploring}\\
- Interactive and Context-Aware Systems in Tourism, 2020
- Context-aware recommender system: A review of recent developmental process and future research direction, 2017,\\ 
- A personalised travel recommender system utilising social network profile and accurate GPS data, 2018\\
- Efficient user profiling based intelligent travel recommender system for individual and group of users, 2019\\
- E-tourism recommender systems: a survey and development perspectives,\\
- The contextual turn: From context-aware to context-driven recommender systems, 2016\\
- \cite{adomavicius2011context,rajaonarivo2019recommendation},\\
- Using POI functionality and accessibility levels for delivering personalized tourism recommendations\\
- Building a model-based personalised recommendation approach
for tourist attractions from geotagged social media data.\\
- Ingrid Christensen, Silvia Schiaffino, and Marcelo Armentano.
Social group recommendation in the tourism domain. Journal
of intelligent information systems, 47(2):209–231, 2016\\
- Kraisak Kesorn, W Juraphanthong, and A Salaiwarakul. Perso-
nalized attraction recommendation system for tourists through
check-in data. IEEE Access, 5:26703–26721, 2017\\
- Lim, Kwan Hui, et al. "Tour recommendation and trip planning using location-based social media: a survey."
Knowledge and Information Systems, pp.: 1-29, 2018.
}

\subsection{Recommender Systems and E-Tourism}
Recommender systems are "a class of information retrieval systems, designed as a set of information resources to filter out just those instances of data that best meet the interests of a particular user" \cite{artemenko2017tourism}. They are usually classified into two categories: memory-based and model-based. \textbf{Memory-based} recommender systems use data of the users or the items and the similarities between them, hence recommending similar items to the ones the user likes (\textit{content-based recommendation}) or recommending items liked by users that are similar to the user (\textit{collaborative filtering}). \textbf{Model-based} recommender systems try to guess how much a user will like an item that they did not consume before, usually through statistical techniques and machine-learning models. In the field of e-tourism, there are many proposed system that utilize both memory-based and model-based techniques, and are called \textit{hybrid recommender systems} \cite{logesh2019exploring}\cite{logesh2019efficient}. Some others just rely on model-based approaches \cite{sun2019building}.

Several research works have been made in the field of context-aware recommender systems in e-tourism. In traditional recommender systems the item score usually is a function of the user and the item:
\begin{equation}
    score: User \ \mathbf{x} \ Item \rightarrow \mathbb{R}
\end{equation}
but in context-aware recommender systems the function is as follow \cite{artemenko2017tourism}:
\begin{equation}
    score: User \ \mathbf{x} \ Item \ \mathbf{x} \ Context \rightarrow \mathbb{R}
\end{equation}

These systems sometimes take into consideration user location \cite{logesh2018personalised},  physical or psychological limitations \cite{santos2019using}, or just weather, time and day \cite{bahramian_abbaspour_claramunt_2017}.

\subsection{Usage of ontologies: Spreading Activation}
Previous work \cite{bahramian_abbaspour_claramunt_2017} integrates the concepts of recommender systems with ontologies at building a context-aware tourism recommender system based on the \textit{spreading activation} algorithm. In its authors' own words: "In spreading activation method, a given concept is represented by a node and has an activation value. A relation among different concepts is represented by a link between nodes and has a weight value. To initialize the algorithm, one or several nodes of a network are activated and these activations spread to the relevant nodes. This process is iterated until a stopping condition (e.g., number of node processed) is reached". The activation value of the node \(v_i\) is computed as follows:
\begin{equation} \label{eq:og_activation}
a_{v_i} = \sum_{v_j \in neighbors(v_i)} w_{v_i, v_j} a_{v_j} 
\end{equation}
where $a_{v_i}$ is the activation value of node $v_i$, $neighbors(v_i)$ is the set of $v_i$'s neighbor nodes and $w_{v_i, v_j}$ is the weight of the link between $v_i$ and $v_j$. This concept will be useful in further sections. 

\subsubsection{Semantic Network}
Behmarian et al. \cite{bahramian_abbaspour_claramunt_2017} extend the information available for a tourism ontology so it can be used with spreading activation algorithm. A \textit{preference} and a \textit{confidence} values are associated to each ontology class. Each link has a weight that represents the degree of relationship between two classes or concepts. An ontology of different context scenarios or \textit{context factors} is linked to the tourism ontology, where the context factors are related to distance to POIs, time and weather information. The activation values of the nodes of the context ontology represent the level of fulfillment based on some measurement. The extended ontology is called \textbf{semantic network}.

\subsubsection{Learning Phase}
Bahmarian et al. \cite{bahramian_abbaspour_claramunt_2017} defined the \textit{preference} and the \textit{confidence} for an ontology class $c$ during learning phase as follow:

\begin{equation} \label{eq:preference}
    pref_c = \frac{\displaystyle \sum_{p \in ancestors(c)}{conf_p pref_p}}
                    {\displaystyle  \sum_{p \in ancestors(c)} {conf_p}}
\end{equation}

\begin{equation} \label{eq:confidence}
    conf_c = \frac{\displaystyle \sum_{p \in ancestors(c)} {conf_p}}{|ancestors(c)|} - \alpha
\end{equation}
where $ancestors(c)$ is the set of ancestors of the ontology class $c$, $pref_c$ is the preference of the class $c$, $conf_c$ is the confidence of the class $c$ and $\alpha$ is the \textit{decrease rate} that will tell how much should decrease the \textit{confidence} at each level. These formulas are applied to each node traversing from the root or sources.

\subsubsection{Recommendation Phase}
To contextualize the recommendation, "\textit{the context factors are used as initial nodes in the spreading activation and transmit the activation flow}" \cite{bahramian_abbaspour_claramunt_2017}, then the activated nodes are recommended. 

The feature of using sets of similar items without using statistical models such as clustering but using ontology classes, and also predicting the user preferences, makes this kind of systems to be hybrid recommender systems.

\subsection{Serendipity} \label{section:serendipity}
An useful goal for a recommender system is to be serendipitous, which accordingly to Kotkov et al. \cite{kotkov2016survey} means that the recommended items must be relevant, novel and unexpected. A relevant item is an item that the user likes, consumes or is interested in; a novel item is an item that the user has never seen or heard about in their life; an unexpected item is an item that significantly differ from the profile of the user. The same work explains the differences between novelty and unexpectedness: "\textit{First, a user may find an item unexpected even if she/he is familiar with the item. Second, to surprise a user, unexpected items have to differ from the user profile more than novel items.}".

Kotkov et al. \cite{kotkov2016survey} define many ways to measure how serendipitous a recommender system is, through measuring first relevance, novelty and unexpectedness of its recommendations. Since a recommender system will probably predict the preference a user has for a specific item, hence predict the relevance of the item, we are mainly focused on novelty and unexpectedness.

There is a metric proposed by Nakatsuji et al. (as cited in \cite{kotkov2016survey}) for novelty that uses the taxonomy classes to which each recommended item belongs and their distance between them, being the minimum distance of a new recommended item $i$ to each already recommended item $j$ the novelty of the item $i$:
\begin{equation}
    nov_{na}(i, u) = \underaccent{j \in I_u}{min}(dis(cls_j, cls_i))
\end{equation}
where $cls_k$ is a class of item $k$ in the taxonomy and $dis(cls_j, cls_i)$ is the distance between classes $cls_j$ and $cls_i$ in the taxonomy. Since our work is focused on the use of an ontology on tourism field, this metric is very helpful.

\subsection{Ontologies for e-tourism algorithm}
