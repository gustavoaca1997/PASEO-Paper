\section{Related Work} \label{section:related-work}

% \textcolor{red}{Clasificar los sistemas de recomedación de acuerdo a: User-centric (manual, automatic); context-aware; ontology-based; used methods (ML, DL, heurísticas, modelos, etc.)\\
% INCLUIR (y otros):\\

% - Efficient user profiling based intelligent travel recommender system for individual and group of users, 2019 NO PDF\\
% - Ingrid Christensen, Silvia Schiaffino, and Marcelo Armentano.Social group recommendation in the tourism domain. Journal of intelligent information systems, 47(2):209–231, 2016\\ NO PDF
% - L Rizaldy Hafid Arigi, Z K Abdurahman Baizal, and Anisa Herdiani. Context-aware recommender system based on ontology for recommending tourist destinations at bandung. Journal ofPhysics: Conference Series, 971:012024, mar 2018. } NO PDF

%\subsection{Recommender Systems} %\label{section:recommender-systems}

The unleashed proliferation  of recommender systems for tourism domain, due to the current development of mobile Internet technology, has fostered a wide categorization of them. 
Depending on the technique used to resolve recommendations, these systems are usually classified into two categories: memory-based and model-based~\cite{bobadilla2013recommender,ebrahim_2012}. \textbf{Memory-based} recommender systems support their  suggestions on similarities among users and their shared items, hence recommending similar items to the ones the user likes (i.e., \textit{content-based recommendation}) or recommending items liked by users that are similar to the user (i.e., \textit{collaborative filtering}). \textbf{Model-based} recommender systems try to guess how much a user will like an item that they did not consume before, usually through statistical techniques and machine-learning models. The recommender systems that use both memory-based and model-based approaches are called \textbf{Hybrid} recommender systems.

In this section, we survey some relevant and recent studies in the tourism domain, classifying them according these three categories and considering aspects related to  how they manage users' preferences and interests, the context awareness, the use of ontologies, and the variability of recommendations. We compare the related work in terms of these criteria and highlight the difference with our proposal.    

\subsection{User information}
Obviously, all recommender systems take into account some information related to the user, such as identification aspects (sex, age, profession, etc.), interest and preferences, or social relations. This information can be obtained from explicit (e.g., questionnaires)~\cite{jannach2020interactive} or implicit (e.g., by analyzing feedbacks in user's social networks) media~\cite{lin2018hybrid}.  

Some proposals are supported on users' interaction to obtain their data. In  %Rajaonarivo et al.
~\cite{rajaonarivo2019rec}, authors model the users' information considering their gender, age category (e.g., kid, adult, or elderly), and preferences, classified as thematic preferences (e.g., museum, theater) and historical preferences (e.g., 12th century), that have to be provided by them through a user interface.  In~\cite{santos2019using}, only health,  physical, and psychological conditions are required  with forms filled by users. 
Users' preferences and feedback are directly assigned by the users in the proposal presented in~\cite{bahramian_abbaspour_claramunt_2017}. As in ~\cite{arigi2018context}, users' interest (i.e., not interested, less interested, interested enough, and not interested) on tourism categories are asked to users. SMARTMUSEUM~\cite{ruotsalo2013smartmuseum}, asks users about the desired duration of a visit to a particular location, the motivation for a visit, and ability to consume the content offered by the system.
 

Other recommender systems extract users' information, mainly their preferences and interests, from available sources,
without asking for explicit interaction.  The work presented in~\cite{kesorn2017personalized}, takes from Facebook basic information (e.g., name, age) and  check-in data (e.g., visited places)  to identify users' interests and preferences.  In~\cite{logesh2019exploring,logesh2018personalised}, 
Users' preferences are deduced from social networks based on opinion mining techniques in~\cite{logesh2019exploring,logesh2018personalised} and on analyzing implicit/explicit feedback~\cite{hidasi2016general}. 


Many other proposals combine both explicit and implicit users' information gathering. SPETA~\cite{garcia2009speta},  supports  its recommendation on the interest and rating  of tourism places that users explicitly provide, and on preferences deduced by analyzing their behavior on social networks. The system presented in~\cite{alonso2012ontology}, takes into account special needs and  context-dependent preferences on tourism sites, directly  provided by users, as well as explicit/implicit feedback on their social networks. 

\subsection{Context-awareness}
In order to improve suggestions, the trend is to consider context aspects that describe a specif situation in a determined moment for a user, including transportation media, weather, time, or even health conditions. 

Many works use context-modeling approaches that mainly consider means of transportation, travel time,  location, or weather~\cite{rajaonarivo2019rec}\cite{bahramian_abbaspour_claramunt_2017}\cite{arigi2018context}\cite{kesorn2017personalized}\cite{logesh2019exploring} \cite{logesh2018personalised}.

Santos et al. \cite{santos2019using} consider physical and psychological limitations of users. \textcolor{red}{Esto es contexto o es información del usuario? Ya está como inf. del usuario en la sección anterior. Considera algo de contexto?}


SMARTMUSEUM~\cite{ruotsalo2013smartmuseum} only considers users' location as contextual information captured by the built-in sensors of mobile devices, such as GPS, accelerometers, and RFID readers. SPETA~\cite{garcia2009speta} also considers users' location extracted from users' mobile devices, but it takes into account current and the history of past locations. It also gathers from  mobile devices other contextual information, such as weather forecast and time.




Hidasi et al. \cite{hidasi2016general} propose a General Factorization Framework for context-aware recommender systems. \textcolor{red}{Qué usa de contexto?}



%The work proposed in~\cite{rajaonarivo2019rec},
%Rajaonarivo et al.
% uses a
%context-modeling approach that considers means of transportation, travel time, and location. LISTO

%Kesorn et al. \cite{kesorn2017personalized} use the time of the day for achieving more suitable recommendations. LISTO

The system proposed in~\cite{alonso2012ontology}, is the only one among the referenced works that considers context-dependent   preferences, as our proposal,  related to weather, day, transport, and special needs.


Sun et al. \cite{sun2019building} use the context of geotagged pictures taken by the user for building the user model. \textcolor{red}{Qué usa de contexto? Además falta referenciarla en la sección de user information}

%Logest et al. \cite{logesh2019exploring} \cite{logesh2018personalised} also consider user context like user location.  LISRO


%Bahmarian et al. \cite{bahramian_abbaspour_claramunt_2017} consider weather, time, and day. LISTO


\subsection{Serendipity}

\textcolor{red}{Agregar por qué serendipia es buena en SR}

Rajaonarivo et al. \cite{rajaonarivo2019rec} model considers previous activity of other users to recommend tours of POIs, thus reducing overspecialization and increasing serendipity, however experiments with this feature were not made.

\textcolor{red}{Se pueden agregar otros ? Pero se tendrían que agregar en las otras sbsecciones que corresponda también}

\subsection{Use of ontologies}

======

Ontology-based  recommender  system  is  an  emerging  trend  in recommender   system~\cite{borras2014intelligent,yochum2020linked}.   Ontology represents   the   concepts   of   a certaindomain   and   their interrelations. In   an   ontology-based   recommender   system, preferences  are  richer  and  more  detailed  than  the  standard keywords-based ones. Moreover, ontology interprets terms with other  terms  according  to  their  semantic  relations.  Also,  its hierarchical  structure  allows  for  an  analysis  of  preferences  at different abstraction levels. Ontology is used to store and exploit the personal preferences of a user and has powerful modeling and reasoning  capabilities.  In  addition,  it  permits  a  high  degree  of knowledge sharing and reuse.


======


\textcolor{red}{Agregar por qué ontología es buena en SR}


Rajaonarivo et al. \cite{rajaonarivo2019rec} export the data used to well known open data platforms before using it; Bahmarian et al. \cite{bahramian_abbaspour_claramunt_2017} based their proposal on spreading activation algorithms on the nodes of an e-tourism ontology.

==============================

RESUMIR DE \cite{bahramian_abbaspour_claramunt_2017}


On  the other hand,  so  far  several  ontology-based  recommender systems  have  been  introduced. However,  these  researches  do  not take  into  account  contextual  situation  in the recommendation process. Moreover, they do not update the contextualized user’s profile based on his feedbacks.

====================================

======NUEVOS ======


 \cite{ruotsalo2013smartmuseum} SMARTMUSEUM ==> The  SMARTMUSEUM  system  is  designed  to  en-hance on-site personalized access to digital cultural her-itage.  The main purpose of the system is to serve theinformation needs of mobile users who are interested incultural heritage.  The system recommends objects onthe basis of a user profile and context information, suchas the physical location and motivation of the user.  Ifthe user is interested in a particular object, the systemcan  retrieve  information  about  that  object  and  relatedcontent  from  the  Web.
 Implicit and explicit desde una GUI (desktop scenario)
 The context service maps contextual data to the con-cepts defined in the ontologies. We look at two types ofcontext information: information that users volunteer toinput – items that are hard to capture via sensors, such as the duration of a visit to a particular location, the user’smotivation for a visit, and ability to consume the content
 offered by the system – and features that can be capturedby the built-in sensors of mobile devices, such as GPSreceivers, accelerometers, and RFID readers, to identifyobjects or the user’s location. ==> ontology-based user profiles and POI ontologies.
 
 


\cite{arigi2018context} context-aware, ontology based
==> user  preferences  of  the  categories  of  tourism  and contextual information such as user locations, weather around tourist destinations and close time of   destination ; ontology as a representation of knowledge on the domain of tourism  category, the  user explicitly  states  an interest in a category of travel destination. There are four levels of user interest, those are not interested, less interested, interested enough and not interested

Arigiet al. \cite{arigi2018context} proposed a context recom-mender based on the ontology system to represent knowledgein the tourism domain. The system recommended touristdestinations by using user preferences of the categories oftourism and contextual information, such as user locations,the weather of tourist destinations, and destination’s closing time.


\cite{garcia2009speta} SPETA ==> which uses knowledge of the user’s current location, preferences,as well as a history of past locations, in order to provide the type of recommender servicesthat tourists expect from a real tour guide. SPETA is a tool created for exploiting the information exposed by user preferences, based on similarities with a currentuser’s social network profile, as well as the contextual information extracted from the user’s location, determined by theuser’s mobile. Therefore, exploiting the processing provided by a pervasive terminal.
real time location, weather forecast, time, user preferences, friend’s recommendations andhistory

The user profile or preferences are constructed in two ways. The first one requires explicit interaction of the user, determining what their interests are, what kind of places they prefer to visit, and the ratings given to attractions. Additionally, ahuge amount of information can be extracted from the social networks they belong to, such as favourite painters, writers, ormusic preferences. Usage of the system provides feedback to the system itself, in other words, user behavior is taken intoconsideration. For example, the type of museums that are always visited by the user or his preferences regarding parksand gardens to spend the evening, etc.The system uses the power of third party social networks to extract information about the user ==> e-tourism ontology (POI)


\cite{alonso2012ontology}  The  system  is  capable  to offer personalized  suggestions to  citizens  and tourist including thosewith  special  needs. he  adaptation  is  concerned  with  several  issues typically  encountered  in  the  representation  of  user  profiles,such   as   cold-start   problem,   context-dependent   preferences representation,   content   filtering   and   user   feedback.   The approach   includes   a   Profile   Manager   for   predicting   user unknown   features   (or   preferences),   reducing   the   need   of querying the  user and expanding the  adaptation possibilities; a Preference Reasoner for handling context-dependent preferences; a Content Manager for creating flexible queries for retrieving and   integrating   the   content   of   heterogeneous information  sources,anda  Feedback  Manager  for interpreting user   feedback   to   refine   the   user   profile   to   progressively enhance the suggestions provided. Context: weather, day, transport, special needs. From explicit way takes only user preferences, it does not keep id information but stereotypes. Feedback implicit or explicit.
Ontology for user preferences and context.


=============================

Table \ref{table:related-work} shows a summary of the survey. Rajaonarivo et al. \cite{rajaonarivo2019rec} is the only work that proposes a context-aware system using ontologies and approaching serendipity, however they did not experiment the serendipitous feature.


\begin{table*}[h!]
    \centering
    \caption{Related work on recommender systems for e-tourism}
    \label{table:related-work}
    \begin{tabular}{|c|c|c|c|c|c|} 
        \hline
        \textbf{Work} & \textbf{Category} & \textbf{Users' information} & \textbf{Context}&\textbf{Ontology}&\textbf{Serendepity} \\\hline \hline 
        & Model-based  & Basic& Heath & User/Tourism & Aging\\    \hline
                &Hybrid &Basic/Preferences & Transportation& Tourism & Spoil \\    \hline
    \end{tabular}
    
    \end{table*}


\begin{table*}[h!]
    \centering
    \caption{Related work on recommender systems for e-tourism}
    \label{table:related-work}
    \begin{tabular}{ |>{\centering\arraybackslash}m{3cm}|>{\centering\arraybackslash}m{1cm}|>{\centering\arraybackslash}m{2.7cm}| } 
        \hline
        \textbf{Works} & \textbf{Type} & \textbf{Approaches} \\
        \hline

        \begin{itemize}
            \item Rajaonarivo et al. \cite{rajaonarivo2019rec}
        \end{itemize} &
        
            Hybrid filtering &

            \begin{itemize}
                \item Context-awareness
                \item Use of ontologies
                \item Serendipity
            \end{itemize}
        
        \\ \hline

        \begin{itemize}
            \item Kesorn et al. \cite{kesorn2017personalized}
            \item Logest et al. \cite{logesh2019exploring}
            \item Logest et al. \cite{logesh2018personalised}
        \end{itemize} &

            Hybrid filtering &

            \begin{itemize}
                \item Context-awareness
            \end{itemize}

        \\ \hline

        \begin{itemize}
            \item Sun et al. \cite{sun2019building}
            \item Santos et al. \cite{santos2019using}
            \item Hidasi et al. \cite{hidasi2016general}
        \end{itemize} &
            
            Model based &
        
            \begin{itemize}
                \item Context-awareness
            \end{itemize}

        \\

        \hline
    \end{tabular}
    \caption{Related work on recommender systems for e-tourism}
    \label{table:related-work}
    \end{table*}

% \subsection{Usage of ontologies: Spreading Activation} \label{section:usage-spreading-activation}
% Previous work \cite{bahramian_abbaspour_claramunt_2017} integrates the concepts of recommender systems with ontologies at building a context-aware tourism recommender system based on the \textit{spreading activation} algorithm. In its authors' own words: "In spreading activation method, a given concept is represented by a node and has an activation value. A relation among different concepts is represented by a link between nodes and has a weight value. To initialize the algorithm, one or several nodes of a network are activated and these activations spread to the relevant nodes. This process is iterated until a stopping condition (e.g., number of node processed) is reached". The activation value of the node \(v_i\) is computed as follows:
% \begin{equation} \label{eq:og_activation}
% a_{v_i} = \sum_{v_j \in neighbors(v_i)} w_{v_i, v_j} a_{v_j} 
% \end{equation}
% where $a_{v_i}$ is the activation value of node $v_i$, $neighbors(v_i)$ is the set of $v_i$'s neighbor nodes and $w_{v_i, v_j}$ is the weight of the link between $v_i$ and $v_j$. This concept will be useful in further sections. 

% \subsubsection{Semantic Network}
% Behmarian et al. \cite{bahramian_abbaspour_claramunt_2017} extend the information available for a tourism ontology so it can be used with spreading activation algorithm. A \textit{preference} and a \textit{confidence} values are associated to each ontology class. Each link has a weight that represents the degree of relationship between two classes or concepts. An ontology of different context scenarios or \textit{context factors} is linked to the tourism ontology, where the context factors are related to distance to POIs, time and weather information. The activation values of the nodes of the context ontology represent the level of fulfillment based on some measurement. The extended ontology is called \textbf{semantic network}.

% \subsubsection{Learning Phase}
% Bahmarian et al. \cite{bahramian_abbaspour_claramunt_2017} defined the \textit{preference} and the \textit{confidence} for an ontology class $c$ during learning phase as follows:

% \begin{equation} \label{eq:preference}
%     pref_c = \frac{\displaystyle \sum_{p \in ancestors(c)}{conf_p pref_p}}
%                     {\displaystyle  \sum_{p \in ancestors(c)} {conf_p}}
% \end{equation}

% \begin{equation} \label{eq:confidence}
%     conf_c = \frac{\displaystyle \sum_{p \in ancestors(c)} {conf_p}}{|ancestors(c)|} - \alpha
% \end{equation}
% where $ancestors(c)$ is the set of ancestors of the ontology class $c$, $pref_c$ is the preference of the class $c$, $conf_c$ is the confidence of the class $c$ and $\alpha$ is the \textit{decrease rate} that will tell how much should decrease the \textit{confidence} at each level. These formulas are applied to each node traversing from the root or sources.

% \subsubsection{Recommendation Phase}
% To contextualize the recommendation, "\textit{the context factors are used as initial nodes in the spreading activation and transmit the activation flow}" \cite{bahramian_abbaspour_claramunt_2017}, then the activated nodes are recommended. 

% The feature of using sets of similar items without using statistical models such as clustering but using ontology classes, and also predicting the user preferences, makes this kind of systems to be hybrid recommender systems.

% \subsection{Serendipity} \label{section:serendipity}
% An useful goal for a recommender system is to be serendipitous, which accordingly to Kotkov et al. \cite{kotkov2016survey} means that the recommended items must be relevant, novel and unexpected. A relevant item is an item that the user likes, consumes or is interested in; a novel item is an item that the user has never seen or heard about in their life; an unexpected item is an item that significantly differ from the profile of the user. The same work explains the differences between novelty and unexpectedness: "First, a user may find an item unexpected even if she/he is familiar with the item. Second, to surprise a user, unexpected items have to differ from the user profile more than novel items.".

% Kotkov et al. \cite{kotkov2016survey} define many ways to measure how serendipitous a recommender system is, through measuring first relevance, novelty and unexpectedness of its recommendations. Since a recommender system will probably predict the preference a user has for a specific item, hence predict the relevance of the item, we are mainly focused on novelty and unexpectedness.

% There is a metric proposed by Nakatsuji et al. (as cited in \cite{kotkov2016survey}) for novelty that uses the taxonomy classes to which each recommended item belongs and their distance between them, being the minimum distance of a new recommended item $i$ to each already recommended item $j$ the novelty of the item $i$:
% \begin{equation}
%     nov_{na}(i, u) = \underaccent{j \in I_u}{min}(dis(cls_j, cls_i))
% \end{equation}
% where $cls_k$ is a class of item $k$ in the taxonomy and $dis(cls_j, cls_i)$ is the distance between classes $cls_j$ and $cls_i$ in the taxonomy. Since our work is focused on the use of an ontology on tourism field, this metric is very helpful.

% % \subsection{Ontologies for e-tourism algorithm}
