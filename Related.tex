\section{Related Work}

\textcolor{red}{Clasificar los sistemas de recomedación de acuerdo a: User-centric (manual, automatic); context-aware; ontology-based; used methods (ML, DL, heurísticas, modelos, etc.)\\
INCLUIR (y otros):\\
- Interactive and Context-Aware Systems in Tourism, 2020 \cite{jannach2020interactive}\\
- E-tourism recommender systems: a survey and development perspectives, 2017, \cite{artemenko2017tourism}\\
-Exploring hybrid recommender systems for personalized travel applications, 2019, \cite{logesh2019exploring}\\
- Interactive and Context-Aware Systems in Tourism, 2020
- Context-aware recommender system: A review of recent developmental process and future research direction, 2017,\\ 
- A personalised travel recommender system utilising social network profile and accurate GPS data, 2018\\
- Efficient user profiling based intelligent travel recommender system for individual and group of users, 2019\\
- E-tourism recommender systems: a survey and development perspectives,\\
- The contextual turn: From context-aware to context-driven recommender systems, 2016\\
- \cite{adomavicius2011context,rajaonarivo2019recommendation},\\
- Using POI functionality and accessibility levels for delivering personalized tourism recommendations\\
- Building a model-based personalised recommendation approach
for tourist attractions from geotagged social media data.\\
- Ingrid Christensen, Silvia Schiaffino, and Marcelo Armentano.
Social group recommendation in the tourism domain. Journal
of intelligent information systems, 47(2):209–231, 2016\\
- Kraisak Kesorn, W Juraphanthong, and A Salaiwarakul. Perso-
nalized attraction recommendation system for tourists through
check-in data. IEEE Access, 5:26703–26721, 2017\\
-
}




\subsection{Recommender Systems}
\subsection{E-Tourism}
\subsection{Spreading Activation}
\cite{bahramian_abbaspour_claramunt_2017} integrate the concepts of recommender systems with semantic web at building a context-aware tourism recommender system based on the \textit{spreading activation} algorithm. In their own words: "\textit{In spreading activation method, a given concept is represented by a node and has an activation value. A relation among different concepts is represented by a link between nodes and has a weight value. To initialize the algorithm, one or several nodes of a network are activated and these activations spread to the relevant nodes. This process is iterated until a stopping condition (e.g., number of node processed) is reached}". The activation value of the node \(v_i\) is computed as follows:
\begin{equation} \label{eq:og_activation}
a_{v_i} = \sum_{v_j \in neighbors(v_i)} w_{v_i, v_j} a_{v_j} 
\end{equation}
where $a_{v_i}$ is the activation value of node $v_i$, $neighbors(v_i)$ is the set of $v_i$'s neighbor nodes and $w_{v_i, v_j}$ is the weight of the link between $v_i$ and $v_j$.

\subsubsection{Semantic Network}
\cite{bahramian_abbaspour_claramunt_2017} extend the information available for a tourism ontology so it can be used with spreading activation algorithm. A \textit{preference} and a \textit{confidence} values are associated to each ontology class. Each link has a weight that represents the degree of relationship between two classes or concepts. An ontology of different context scenarios or \textit{context factors} is linked to the tourism ontology, where the context factors are related to distance to POIs, time and weather information. The activation values of the nodes of the context ontology represent the level of fulfillment based on some measurement. The extended ontology is called \textbf{semantic network}.

\subsubsection{Learning Phase}
\cite{bahramian_abbaspour_claramunt_2017} defined the \textit{preference} and the \textit{confidence} for an ontology class $c$ during learning phase as follows:

\begin{equation} \label{eq:preference}
    pref_c = \frac{\displaystyle \sum_{p \in ancestors(c)}{conf_p pref_p}}
                    {\displaystyle  \sum_{p \in ancestors(c)} {conf_p}}
\end{equation}

\begin{equation} \label{eq:confidence}
    conf_c = \frac{\displaystyle \sum_{p \in ancestors(c)} {conf_p}}{|ancestors(c)|} - \alpha
\end{equation}
where $ancestors(c)$ is the set of ancestors of the ontology class $c$, $pref_c$ is the preference of the class $c$, $conf_c$ is the confidence of the class $c$ and $\alpha$ is the \textit{decrease rate} that will tell how much should decrease the \textit{confidence} at each level. These formulas are applied to each node traversing from the root or sources.

\subsubsection{Recommendation Phase}
To contextualize the recommendation, "\textit{the context factors are used as initial nodes in the spreading activation and transmit the activation flow}" in the work of \cite{bahramian_abbaspour_claramunt_2017}.

\subsection{Ontologies for e-tourism algorithm}
